%!TEX root = ../Thesis.tex
\chapter{Applications}
\label{chap:applicatons}
In the previous chapters we learned that other than viewer's discomfort, reduction of contrast, reduction of stereo acuity and reduction of overall image aesthetics, the crosstalk in stereoscopic screens also has a substantial effect on the observed depth of the desperate objects (i.e. the objects not at the plane of focus). The general rule is that the observed depth for any object not lying on the plane of focus\index{POF} will tend to fall back to the POF (loosing depth) as the crosstalk level increases or as the disparity of an object increases. The effect seemed to be more pronounced for objects with narrow width compared to the objects with comparatively wider width. The reason being that the ghost separation (which causes the confusion for HVS) for thin objects is larger for any given disparity. We also learned that contrary to our intuition, the crosstalk in an automultiscopic screen has little to no effect on the perceived depth for objects of any width for even the most extreme level of crosstalk (14\%). The main reason why crosstalk  did not affect the observed depth might be that in an automultiscopic case, there are two similar ghosts present on each side of the object in both retinal images. This makes the ghost-object combination in both eyes very similar and the HVS might consider them as the same object. Hence there is nothing to confuse the HVS in estimating the correct disparity between two retinal image patches.

We also learned that to the best of our knowledge, the current literature seems to agree with the theory that HVS resolves the estimated disparities of objects in a binocular scene by computing the cross-correlation of several patches located at different positions in between two perspective retinal images. Finally, we learned that in order to reduce the viewers crosstalk, most of the current state of the art techniques rely upon preprocessing the stereo images in such a way that when viewed on a stereoscopic or automultiscopic screen, the system crosstalk added images resembles the actual images. Most techniques subtract the pre-calculated system light intensity leakage between views from the actual images in addition to using some perceptual measured in order to mask the ghosting that still persists due to the limited dynamic range of the screens.

Observed depth reduction due to crosstalk is a serious problem which might alter or degrade the artistic viewing experience for a scene. Just like we have image quality metric e.g. SSIM \cite{wang2004image}, MSE\cite{ wiki:MSE} or VDP\cite{mantiuk2004visible} etc that are used to calculate the effects of noise in an image on its perception by a human observer, it would be useful if we have a quality metric that is able to predict the observed depth of objects of interest in a 3D scene based on the crosstalk level of the display along with the dimensions and disparity of the object. Once this prediction is available, the scene artists can then alter the theoretical depths of the objects in order to bring the observed depths as close to the original as possible. For this to work however, we should have a good understanding of how the HVS estimates the depth from disparity between two retinal images. Moreover, reducing the viewers crosstalk for any given display is also as important if not more for a better viewing experience. During our research, we learned that the most promising technique to achieve that used complex optimizations to be performed in order to pre-process the images\cite{van2011perceptually}. It would be helpful if the similar effects can be achieved without the mentioned complex and time consuming optimizations.

In this chapter, we will look into a proposed modified HVS depth from disparity resolution model that provides good approximation (not ideal but promising) to the observed depths obtained from the experiment results discussed in the previous chapter. We will also propose two new crosstalk reduction techniques along with there pros/cons and results.

\section{Observed depth prediction}

Filippini et al \cite{filippini2009limits} proposed a model that simulated the disparity estimation of the HVS via disparity. In short, this model computes the local cross-correlation that is defined by eq \ref{eq:banks_ccr}
\begin{equation}
c(\delta_x) = \frac{ \sum\limits_{(x,y) \in W_L} [(L(x,y) - \mu_L)(R(x-\delta_x, y) - \mu_R)] }{\sqrt{\sum\limits_{(x,y) \in W_L}(L(x,y) - \mu_L)^2} \sqrt{\sum\limits_{(x,y) \in W_R}(R(x-\delta_x, y)- \mu_R)^2}}
\label{eq:banks_ccr}
\end{equation}
where
\begin{equation}
W_L \:=\: W_R \:=\: e^{-\left(\frac{x^2}{2\sigma_x^2} \:+\: \frac{y^2}{2\sigma_y^2}\right)}
\label{ccr_windows}
\end{equation}
are anisotropic Gaussian windows (patches) in the left and the right stereo images. The window $W_L$ is fixed in one image while $W_R$ is displaced horizontally while keeping the vertical position constant. For each displacement, $c(\delta_x)$ is computed that results in a value of +1 if the patches are perfectly correlated or -1 if the patches have no correlation at all or a value in between +1 and -1. The `$\delta_x$' for which the maximum cross-correlation is attained is considered to be matching pixel (or patch) in the right image w.r.t the left image. The process is repeated for all the pixels in the left image. This model was tested on random dot stereogram consisting of saw-toothed corrugation and the resulting computed disparities for each part of the stereogram matched the results obtained via experiments on human observers. The random dot stereograms used however were devoid of any crosstalk.

We tested the same model on our stimuli (Chapter 4) and found that this model always resulted in the correct disparity of the objects between a stereo image pair. I.e the crosstalk had no effect on the estimated disparity. This did not match the results we obtained for the human observers. The reason being that even thought the areas on the images where the ghosts are present will show some positive correlation, the maximum correlation is always obtained at the location where the actual object is located.

Hence it is clear that the HVS does not simply estimate the depth from disparity via local cross-correlation only and some kind of pre-processing or post-processing is performed. It is believed that the HVS, while matching patches between retinal images, prefers lower disparity over higher ones. This means that for any patch in say left eye retinal image, if the HVS finds two matching patches in the right image, it will choose the one that is closest to the location of the left image patch. Recall from the previous chapter that in case of stereo displays with some level of crosstalk, for a desperate object in the the left image, the same object is located at some horizontal disparity `d' where as the ghost is located exactly at zero disparity (same location as the location of the object in the left image). Computing a cross-correlation in this case would result in some positive correlation because of the ghost at zero disparity and a highly positive correlation at disparity `d'. To match the results from human observers, the disparity estimator model should
\begin{itemize}
\item{Select a disparity `d' based on where the cross-correlation results in a maximum value.}
\item{Estimated disparity `d' should shift towards zero disparity gradually as the actual disparity increases some threshold.}
\item{Estimated disparity should be estimated as 0 if the crosstalk level increases some threshold.}
\end{itemize}
Hence we have to weigh the resulting cross-correlation profile for all disparities. We observe that weighing the said cross-correlation profile with a Gaussian centered at the location of object in the left eye image fulfills the above mentioned requirements as seen in fig \textcolor{red}{insert the figure for }.

In our disparity estimator model, we ...


% \section{Depth prediction application}

% \section{Crosstalk mitigation}
% \subsection{Proposed optimizations}
% \subsection{Unsharp masking in view domain}
% \subsection{Iterative subtraction}
