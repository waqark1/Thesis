%!TEX root = ../Thesis.tex
\chapter{Related work}
\label{chap:relatedWork}
It is widely believed that crosstalk is stereoscopic or automultiscopic screen results in reduced visual comfort, reduced disparity range, reduced contrast and most importantly, reduction of perceived depth. However, the literature is not thorough about how the  crosstalk effects the perceived depth. Wilcox et al \cite{wilcox2003determinants} performed a large scale experiment in which 77 observers were shown a small stereo footage in two commercial large-formate 3D theaters. The typical stereo degradations in theaters that were considered were crosstalk (ghosting), brightness, and the other qualities of 3D. The test subjects were asked to evaluate which one those degradations affected their experience the most. 75\% of the subjects reported that ghosting was the most prominent in degrading their viewing experience.
\textcolor{red}{Write something more.}

Below is a review of some of the work that has previously been done in this area.

\section{Effects of crosstalk on perceived depth}
K.C.Huang et al \cite{huang2003crosstalk} performed intensive experiments in order to obtain a threshold for the system crosstalk that on average will not mitigate the effects of the depth from disparity. Since the viewer crosstalk is dependent upon the system crosstalk and the local contrast of the desperate object in the image, a uniform level of crosstalk (same crosstalk all over the screen which does not change with time) should not have the same effect on all kinds of stereo images. This is because the HVS sensitivity to detect the change in luminance follows Webber's law \cite{webber}. Which means that compared to areas of the image with high local contrast, the HVS is more tolerable towards crosstalk where the local contrast is low. Based on this idea and the fact that the HVS also uses monocular cues to estimate depth, they performed experiments using a Wheatstone setup (similar to ours) and a set of images with various contrasts and disparities to finally propose that 10\% system crosstalk is the maximum crosstalk that will mostly not affect the depth estimation via disparity. We observed in our experiments that even though above 10\% of crosstalk heavily degrades the perceived depth, it still does not result in total depth loss due to disparity. On our high contrast images, we determined that crosstalk level of greater than 16\% usually resulted in the total loss of depth.

Ghosted images due to crosstalk as seen in figure \textcolor{red}{make figure somewhere} can be seen as locally decreasing the contrast of the desperate object. Rohaly et al \cite{rohaly1999effects} determined the effects that contrast exhibits on the stereoscopically perceived depth. She concluded that decrease in contrast made the objects (crossed and uncrossed) appear to be farther from the viewers. Which means that the crossed objects with decreased contrast appeared to be moving towards the horoptor whereas the uncrossed objects appeared to be moving away from the horopter. She also found that monocular contrast reduction amplified this effect more compared to the contrast reduction in both eyes \textcolor{red}{Mention this in the discussion of the results}. It was also argued that the contrast exerts its effect before or at the extraction of depth.

The work that is most relevant to our work was performed by Tsirlin et al. They quantified the amount of depth loss due to crosstalk on objects of various dimensions and disparities. In 2011 \cite{tsirlin2011effect}, they performed experiments where the test subjects were asked to specify the observed depth at various depths of a stimulus (a rectangular structure) the width of which was chosen to be such that at any disparity, the ghost would not be completely separated. The luminance of the structure was set to be the maximum (white) on a completely black background. The subjects reported the observed depth via a slider bar located under the stimulus without any reference. \textcolor{red}{insert the stimulus and the graph here.}. They observed that the perceived depth decreased with the increase in crosstalk as well as increase in disparity. In the second part of the experiment, they observed the effect of crosstalk on binocular occlusion and found out that effect of crosstalk on the perceived depth due to occlusion was even more swear. In 2012 \cite{tsirlin2012effect}, they performed similar experiments to observe the effects of crosstalk on thin structures where the ghost is always completely separated from the stimulus (which is usually the case with vertical thin structures present at a significant distance from the plane of focus). Again they found that observed depth degrades significantly as the disparity or the crosstalk level was increased. However this time the degradation was observed to be higher than the case where the ghosts always overlapped the stimulus. One problem with these experiments is that the stimuli are presented as a uniformly luminant objects on a black background where there is no other depth cue present. This is usually not the case with the complex scene that we typically observe in 3D movies. We observed in our test experiments that in such scenes the depth of the stimuli was extremely hard to perceive even at moderately high disparities. Tsirlin also observed that the perceived depth degraded with disparity even in the base case i.e. the case where crosstalk was set to zero. This is not usually the case in complex scenes. Secondly, there was no reference used and the subjects were simply asked to report the observed depth via a slider bar that was controlled with a mouse. This `rating' setup can be problematic because it can not be guaranteed that all the subjects were rating the same depth equally \textcolor{red}{write something better}. And finally, the relation of the stimulus width to the perceived depth at different disparities was not determined which we think is also necessary \textcolor{red}{why?}.

Tsirlin et al \cite{tsirlin2012crosstalk} again performed similar experiments but this time with complex natural scenes. The subjects were shown a complex scenes with the objects of interest marked by arrows. Later, they were shown the exact same images with crosstalk induced and were asked to rate the depth difference they observed between those two objects. The results again concluded that the perceived depth difference decreased as the theoretical depth difference between the objects of interest and the crosstalk increased.\textcolor{red}{write some criticism}.

\section{Depth resolution mechanisms in HVS}
\subsection{Banks and cormacks work}


\section{Reduction of crosstalk}

\subsection{Stereoscopic screens}
\subsubsection{Subtractive approaches}
\subsubsection{Perceptual optimization}
\subsubsection{Temporal approach}

\subsection{Automultiscopic screens}
\subsubsection{Inverse filtering}
\subsubsection{Subtractive reduction}
\subsubsection{Sub-pixel optimization}
\subsubsection{Low Pass filtering}