%!TEX root = ../Thesis.tex
\chapter{Introduction}
\label{chap:intro}

Ever since its commercial introduction to cinema in 1922, the Stereoscopic approach to 3D content presentation has waxed and waned in popularity over time. Commercial stereoscopic 3D displays like any other practical system are not devoid of imperfections. One of these imperfection know as crosstalk is the inability to separate the different views completely for each eye and hence dim copies (ghosts) of the unintended images are seen along with the intended images. Crosstalk is present in all commercially available 3D displays (LCD TV's and cinemas) and contributes heavily to the viewers discomfort in the forms of reduced overall image quality and reduced image contrast. More importantly crosstalk also affect the HVS stereopsis. Unintended depth edges conflict with the intended depth edges, thereby hindering the proper fusion of the stereo images \cite{tsirlin2012effect}. These conflicts can result in reduce observed depth. Thanks to today's technologically impressive and immensely lucrative 3D movie industry, TV and games, the 3D industry has been seeing a sharp rise in popularity. This increase in popularity has once again motivated the researchers to give attention to imperfections in commercial stereoscopy and their effects on the viewers. Current methods for crosstalk compensation or `deghosting' typically involves subtracting the unintended ghosts from the intended image before being displayed. This technique however either reduces the overall image contrast or does not remove the ghosting completely in high contrast regions of the images.

\section{Contribution}



\section{Contributions}
Experimentation, mitigation, HVS model.


% \section{List of Commonly used abbreviations}

