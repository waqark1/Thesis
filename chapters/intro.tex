%!TEX root = ../Thesis.tex
\chapter{Introduction}
\label{chap:intro}

Ever since its commercial introduction to cinema in 1922, the stereoscopic approach to 3D content presentation has waxed and waned in popularity over time. Stereoscopy consists of presenting a different perspective image separately to each eye tricking the Human Visual System\index{HVS} (HVS) to see in 3D on a 2D planer flat screen. Current commercial stereoscopic 3D displays, like any other practical system, are not devoid of imperfections. One of these imperfections known as crosstalk, is the inability to separate the different views completely for each eye. This means that some portion of the light from the image intended for one eye leaks to the other eye (and vice versa) that results in dim copies (ghosts) of the unintended images seen along with the intended images. Crosstalk is present in all commercially available 3D displays (LCD TV's and cinemas) and contributes heavily to the viewer's discomfort in the forms of reduced overall image quality and reduced image contrast. More importantly crosstalk also affects the stereopsis of HVS. Unintended depth edges conflict with the intended depth edges, thereby hindering the proper fusion of the stereo images \cite{tsirlin2012effect}. These conflicts can result in reduced observed depth. Thanks to today's technologically impressive and immensely lucrative 3D movie industry, TV and games, the 3D industry has been seeing a sharp rise in popularity. This increase in popularity has once again motivated researchers to give attention to imperfections in commercial stereoscopy and their effects on viewers. Current methods for crosstalk compensation or `deghosting' typically involve subtracting the unintended ghosts from the intended image before being displayed. This technique, however, either reduces the overall image contrast or does not remove the ghosting completely in high contrast regions of the images.


\section{Contribution}

Although the effects of crosstalk on viewer's observed quality and visual discomfort have been thoroughly studied in the past \cite{wilcox2003determinants}, its effects on the depth perception have received little attention. Since the basic purpose of stereoscopy is to display a 3D scene consisting of different objects located at different depths, any undesired variation in the perceived depths can severely hamper the aesthetics of the observed scene. Thus the main focus of this thesis is to better understand the effects that crosstalk can have on the depth perception on various types of images and 3D display technologies. Previously, some  work has been done concluding that crosstalk in general reduces the perceived depth \cite{tsirlin2012effect}, i.e. objects that are distant from the plane of fixation\index{POF} (POF) will tend to fall back to the POF. The user studies that resulted in that conclusion were performed on monochromatic stimuli where disparity was the only cue for depth. The natural scenes usually have a lot more depth cues and hence we hypothesized that the effects of crosstalk might be different for complex stimuli. In order to assess the effects of crosstalk on the viewer's depth perception, we firstly performed experiments on stimuli of various dimensions and shapes that were rendered images of a 3D scene containing all possible depth cues. Further, we found that there have been no studies analyzing the effects of crosstalk on the depth perception in an automultiscopic displays (glass free 3D display). We performed another set of experiments in order to analyze if the effects of crosstalk on depth perception in automultiscopic displays is any different.

We also propose a modified HVS disparity estimation model inspired by the model of Banks et al \cite{filippini2009limits}, that helps understand why the crosstalk results in reduced depth of objects that are located away from the plane of fixation. Finally, we propose and test some new techniques for reducing crosstalk in automultiscopic displays.

\section{Structure}

The thesis is assembled as follows. Chapter 2 discuss some background knowledge required in order to fully understand subsequent chapters. This included information about how the HVS works according to the literature, a formal definition of crosstalk and a general idea of how different 3D displays work along with the nature of their crosstalk. Chapter 3 covers some of the previous research that has been performed in order to understand the effects of crosstalk on perceived depth, the possible explanation for this effect, and different techniques that are used in order to mitigate the effect of crosstalk via image preprocessing. Chapter 4 discusses in detail the experiments we performed to quantify how the depth is degraded in both stereoscopic and automultiscopics screens, along with their results. In Chapter 5, we discuss our proposed modified HVS depth via disparity resolution model that takes the effects of crosstalk into account as well. We also discuss our proposed crosstalk reduction techniques. Finally, in Chapter 6 we will look into some phenomenon that still needs explanation and where the future research could be headed in order to get a better understanding of them.


% \section{List of Commonly used abbreviations}

