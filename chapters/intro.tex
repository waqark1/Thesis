%!TEX root = ../Thesis.tex
\chapter{Introduction}
\label{chap:intro}

Ever since its commercial introduction to cinema in 1922, the Stereoscopic approach to 3D content presentation has waxed and waned in popularity over time. Commercial stereoscopic 3D displays like any other practical system are not devoid of imperfections. One of these imperfection know as crosstalk is the inability to separate the different views completely for each eye and hence dim copies (ghosts) of the unintended images are seen along with the intended images. Crosstalk is present in all commercially available 3D displays (LCD TV's and cinemas) and contributes heavily to the viewers discomfort in the forms of reduced overall image quality and reduced image contrast. More importantly crosstalk also affect the HVS stereopsis. Unintended depth edges conflict with the intended depth edges, thereby hindering the proper fusion of the stereo images \cite{tsirlin2012effect}. These conflicts can result in reduce observed depth. Thanks to today's technologically impressive and immensely lucrative 3D movie industry, TV and games, the 3D industry has been seeing a sharp rise in popularity. This increase in popularity has once again motivated the researchers to give attention to imperfections in commercial stereoscopy and their effects on the viewers. Current methods for crosstalk compensation or `deghosting' typically involves subtracting the unintended ghosts from the intended image before being displayed. This technique however either reduces the overall image contrast or does not remove the ghosting completely in high contrast regions of the images.


\section{Contribution}

The main focus of this thesis is to better understand the effects that crosstalk can have on the perceived depth magnitude. Previously, some work has been done \cite{tsirlin2012effect} concluding that crosstalk in general reduces the perceived depth via disparity estimation by HVS i.e. objects that are distant from the plane of focus will tend to fall back to the POF. This magnitude of depth degradation increases as the disparity or the crosstalk level increase. The user studies were performed on monochromatic stimuli where disparity was the only cue for depth. The natural scenes usually have a lot more depth cues and hence we hypothesized that the effects of crosstalk might be different for complex stimuli. We in this thesis performed similar but more robust experiments on stimuli that were rendered images of a 3D scene with all possible depth cues included. Ghosting on automultiscopic screens behave differently than stereoscopic ones hence we performed experiments for both stereoscopic and automultiscopic displays in order to see if the effect on perceived depth is different or not.

We also propose a modified HVS disparity estimation model which is inspired by \cite{filippini2009limits} that helps understand why the crosstalk results in reduced depth of objects that are located away from the fixation plane. And finally, we propose and test some new ideas of reducing crosstalk in automultiscopic screens.

\section{Structure}

The thesis is assembled as follows. Chapter 2 discuss some background knowledge that are required in order to fully understand the later chapters. This include information about how the HVS according to the literature works, formal definition of crosstalk and a general idea of how different 3D displays work along with their nature of crosstalk. Chapter 3 covers some of the previous research that has been performed in order to understand the effects of crosstalk on perceived depth, the possible explanation for this effect and different techniques that are used in order to mitigate the effect of crosstalk via image preprocessing. Chapter 4 discusses in detail the experiments we performed to quantify how the depth is degraded in both stereoscopic and automultiscopics screens along with their results. In Chapter 5, we discuss our proposed modified model that might mimic the depth estimation of HVS along with our proposed crosstalk reduction techniques. Finally in chapter 6 we will look into some phenomenons that still needs explanation and how the future research should be headed in order to get a better understanding of them.


% \section{List of Commonly used abbreviations}

