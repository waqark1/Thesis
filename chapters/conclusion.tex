%!TEX root = ../Thesis.tex
\chapter{Conclusion}
\label{chap:Conclusion}

\section{Summary}
In this work, we show for the first time, how the crosstalk affects the perceived depth in automultiscopic displays. In addition to that, we also show how the perceived depth is affected for natural stimuli of different dimensions and geometry in stereoscopic displays. From our experiments we verified that in stereoscopic environment where each object has one ghost located on either side, crosstalk had an adverse effect on the perceived depth magnitude. I.e., the perceived distance of the objects from the fixation point appeared to be lower. This loss of depth in general, is directly proportional to the crosstalk level and the disparity of the object, i.e., higher levels of crosstalk or disparity resulted in a greater loss of observed depth. This loss of depth is more severe for thin objects rather than thick ones or ones with complex geometry. In an automultiscopic environment however, where two copies of identical ghosts are located on each side of the object, we found no significant loss of depth. The reason for this might be the identical nature of the ghost-object combination in both eyes, the symmetry of the configuration leading to a maximum cross-correlation at the real depth disparity. 

On the contrary, in the stereoscopic case, the HVS has to make a confused decision between matching the object in one retinal image to either the ghost ( located at zero disparity) or the original object (located at actual disparity) in the other retinal image. After the experiments, a debriefing session was held with the test subjects regarding the overall experience. Even thought human observers reported no loss of perceived depth in the automultiscopic case, they did report that having to deal with a greater number of ghosts added to their viewing discomfort more than the stereoscopic case. Also from our experiments for the stereoscopic environment, we found that depth loss due to crosstalk was negligible for all stimuli as long as the crosstalk levels were kept under 3\%. Due to this we predict that crosstalk level below 3\% should not have any effect of the perceived depth of the objects when viewed on a stereoscopic display.

In order to predict the viewer's observed depths of stimuli when presented on a 3D display, understanding how the HVS resolved for disparity between two retinal images is important. Hence, a faithful HVS model is required. Current HVS disparity estimation model, that estimates the disparity by computing the cross-correlation of patches between two retinal images works well in estimating the disparity of stimuli when no crosstalk is present. It however, fails to correctly estimate the disparities when crosstalk is added to the stimuli. In this work, we propose a modified HVS disparity estimation model that weighs the computed cross-correlation between patches with respect to the crosstalk level and the disparity appropriately. This way, the HVS model estimates the disparity of stimuli that is closer to the actual human observed disparities/depths we showed in our experiments.

\section{Future work}

From our experiments it is clear that crosstalk affects the HVS's stereopsis and, in order for these effects to be negligible, the crosstalk levels in a stereoscopic display should be no more than 3\%. We also know that 5\% crosstalk in stereoscopic displays is enough to induce viewing discomfort. However, we do not know what the minimum threshold for crosstalk in an automultiscopic display should be. A user study would be beneficial to find this limit. In addition to that, the effect of crosstalk on percieved depth in an automultiscopic display, when viewed from a non-sweet spot position should also be analyzed. This analysis should be carried out for both geometrically symmetric and non-symmetric stimuli. More importantly, it is imperative to fully understand how the HVS resolves depth information from a 3D scene. This will not only help us better predict the observed depth of a scene when viewed through a stereoscopic display, but will also help us improve the current crosstalk reduction techniques or help us come up with new ones. Thus, work in more comprehensive models of depth perception is required.

