%!TEX root = ../Thesis.tex
\chapter{Conclusion}

\section{Summary}

From our experiments we verified that in stereoscopic environment where each object has one ghost located on either side, crosstalk had an adverse effect on the perceived depth magnitude. The perceived distance of the objects from the fixation point appeared to be lower. This loss of depth in general, was directly proportional to the crosstalk level and the disparity of the object i.e. higher levels of crosstalk or disparity resulted in a greater loss of observed depth. We also figured out that that this loss of depth is more severe for thin objects rather than thick ones or ones with complex geometry. In automultiscopic scenario however, where two copies of identical ghosts were located on each side of the object, we found that the human observers reported no significant loss of depth. The reason for this might be the identical nature of the ghost-object combination in both eyes fooling the HVS into considering it as one object located at some distance from the fixation point. On the contrary, in stereoscopic case, the HVS has to make a confused decision between matching the object in one retinal image to either the ghost (zero disparity) or the original object (actual disparity) in the other retinal image. Even thought human observers reported no loss of perceived depth in automultiscopic case, they did report that having to deal with greater number of ghosts added to their viewing discomfort more than the stereoscopic case. Also from our experiments we found that depth loss due to crosstalk in stereo had a minimum threshold of 3\%. Due to this we predict that crosstalk level below 3\% should not have any effect of the perceived depth of the objects in the scene.
\pagebreak
\section{Future Work}

From our experiments it is clear that crosstalk affects the HVS stereopsis and hence the crosstalk levels in 3D displays should be no more than 3\%. We also know that as less as 5\% crosstalk in stereoscopic displays is enough to induce viewing discomfort. However we do not know what the minimum threshold for crosstalk in an automultiscopic display should be. Perhaps a user study is required in order to find this limit. More importantly, it is imperative to fully understand how the HVS resolves depth information from a 3D scene. This will not only help us better predict the observed depth of a scene when viewed through a stereoscopic display but may also help us improve the current crosstalk reduction techniques or help us come up with new (completely different) ones.

