%!TEX root = Thesis.tex
\addtotoc{Abstract}  % Add the "Abstract" page entry to the Contents
\abstract{
\addtocontents{toc}{\vspace{1em}}  % Add a gap in the Contents, for aesthetics
\emph{Stereoscopic and automultiscopic displays suffer from crosstalk, that is undesired effect which greatly reduces image quality, viewer comfort and distort the perception of depth. Previously, only a limited work has been done on understanding the relation between crosstalk and the perceived depth with respect to stimuli of different nature in stereoscopic displays. To the best of our knowledge, no such work has been done for automultiscopic displays. Moreover, most of the previous work is carried using simple monochromatic scenes. Since the human visual system uses numerous cues other than disparity to estimate the depth of an object in a stereo scene, monochromatic scenes are poor choice for understanding the aforementioned effects. In this work, we perform experiments to better understand the effects that crosstalk might have on the perceived depth of stimuli in both, stereoscopic and automultiscopic displays. In order to obtain an accurate understanding, we perform experiments on rendered stimuli of natural scenes consisting of objects of various geometries. The model for human visual system's depth estimation via disparity as provided by the current literature fails to justify why and how the perceived depth is affected by the crosstalk. Based on the result of our experiments, we propose a modified human visual system's depth estimation model that, while estimating the viewer's observed depth of a stereoscopic stimulus, takes the ghosting in consideration as well. Finally, some improved techniques for compensation of crosstalk in automultiscopic displays are proposed.}
}
\clearpage